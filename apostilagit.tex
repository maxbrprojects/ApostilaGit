\documentclass[12pt,a4paper]{report}

% Usando UTF-8
\usepackage[utf8]{inputenc}

% Para usar codigos de programas
\usepackage{listings}
\lstset{language=bash}

% Renomeando o "Abstract" para "Resumo"
\renewcommand{\abstractname}{Resumo}
\renewcommand{\contentsname}{Sumário}
\renewcommand{\chaptername}{Capítulo}

\begin{document}

\title{Apostila de introdução à controle de versões e Git}
\author{Evaldo Junior Bento}

\maketitle

\begin{abstract}
Pequenas e grandes equipes de desenvolvimento de \textit{software} sofrem
com problemas de versionamento de código fonte. Quem fez? O que fez? Quando
fez? Quais linhas foram alteradas? Estas são perguntas comuns no dia a dia de
equipes que não usam sistemas de controle de versão.

Esta apostila foi desenvolvida para ajudar iniciantes e interessados em
sistemas de controle de versões à entender os conceitos de controle de versões
e por que fazê-lo. O \textbf{Git}, desenvolvido por Linus Torvalds, criador do
\textit{kernel} Linux, é o software de controle de versões usado nesta
apostila para exemplificar os conceitos.
% Preciso melhorar este texto depois =)
\end{abstract}

\tableofcontents

\chapter{Introdução}
    Desenvolver \textit{software} sem utilizar um sistema de controle de versões...

\chapter{Fluxo comum de trabalho em um projeto de software}
    O fluxo "normal" de trabalho, em um projeto de \textit{software}, pode ser
    resumido em:
    \begin{itemize}
        \item Pegar o código atual;
        \item Editar;
        \item Salvar;
        \item Devolver para o centralizador.
    \end{itemize}
    Até aqui, tudo bem. O desenvolvedor abre o código fonte, faz seu trabalho e
    então devolve o resultado para uma base central. Em projetos individuais ou
    de equipes bem pequenas, como duplas ou trios, esse método pode até
    funcionar, mesmo assim esta não é uma boa solução.
    
    \section{Os problemas começam}
        Imagine alterar um arquivo, colocar em produção e depois de um tempo ver
        que suas alterações simplesmente desapareceram. Isso pode acontecer quando
        outro desenvolvedor também fizer alterações no mesmo arquivo e enviar as
        suas alterações após o primeiro, sem antes verificar se os arquivos do
        projeto continuavam iguais aos que ele pegou antes de alterar.
        
        Veja o tempo que se perde por ter que verificar os arquivos antes de
        cada atualização. E o tempo maior ainda por ter que refazer algo que já
        havia sido feito, fora a frustração e o desanimo que esse tipo de
        situação geralmente causa.
    
    \section{A solução aparece no horizonte}
        E se existisse algum tipo de software que ajudasse a verificar os
        arquivos, saber quem alterou, que linhas alterou e quando alterou?
        
        Sim, isso seria muito legal. Mas espere, isso existe sim! São os
        Sistemas de Controle de Versão\footnote{SCM - Source Code Management,
        em inglês.}.

\chapter{Controle de versões}
    O que é?

\chapter{Git}
    Desenvolvido por Linus Torvalds

\chapter{Iniciando um projeto}
    Para iniciar um novo projeto, crie um diretório, ou use um já existe,
    inclusive onde já existem arquivos, e use, neste diretório, a opção
    \textbf{init} do Git:
    \begin{lstlisting}
        $ mkdir projeto
        $ git init
    \end{lstlisting}
    Agora crie os arquivos ou trabalhe no seu projeto, normalmente. Até este
    momento o Git ainda não sabe da existência dos arquivos do projeto, para
    que os arquivos sejam adicionados ao controle de versões use a opção
    \textbf{add} do Git:
    \begin{lstlisting}
        $ git add arquivo
    \end{lstlisting}
    Também é possível adicionar todos os arquivos novos/alterados de uma vez
    usando a opção \textbf{ponto}:
    \begin{lstlisting}
        $ git add .
    \end{lstlisting}
    Depois de adicionar os arquivos é necessário fazer o \textit{commit}%
    \footnote{Commit: É uma confirmação do que foi feito.}. % Preciso melhorar essa descrição do commit...
    Para fazer o commit use a opção \textbf{commit} do Git:
    \begin{lstlisting}
        $ git commit -m "Mensagem descrevendo o que foi feito"
    \end{lstlisting}
    A opção \textbf{-m} usada no comando acima permite adicionar uma pequena
    mensagem de \textit{commit} diretamente na linha de comando. Caso essa
    opção não seja informada será aberto o editor de textos padrão para que a
    mensagem seja digitada. Em geral o editor usado é o \textbf{Vim} ou o
    \textbf{Nano}. Optar por não usar o \textbf{-m} é uma boa escolha quando
    se deseja escrever mensagens de \textit{commit} maiores e mais detalhadas.
\end{document}
