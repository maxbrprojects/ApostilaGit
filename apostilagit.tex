\documentclass[12pt,a4paper]{report}

% Usando UTF-8
\usepackage[utf8]{inputenc}

% Renomeando o "Abstract" para "Resumo"
\renewcommand{\abstractname}{Resumo}
\renewcommand{\contentsname}{Sumário}

\begin{document}

\title{Apostila de introdução à controle de versões e Git}
\author{Evaldo Junior Bento}

\maketitle

\begin{abstract}
Esta apostila foi desenvolvida com a ideia de ajudar iniciantes e interessados
em sistemas de controle de versões. O \textbf{Git} é o software utilizado
para fazer os exemplos.
% Preciso melhorar este texto depois =)
\end{abstract}

\tableofcontents

\section{Introdução}
Desenvolver \textit{software} sem utilizar um sistema de controle de versões...

\section{Controle de versões}
O que é?

\section{Git}
Desenvolvido por Linus Torvalds

\section{Fluxo comum de trabalho em um projeto de software}
    O fluxo "normal" de trabalho, em um projeto de \textit{software}, pode ser
    resumido em:
    \begin{itemize}
        \item Pegar o código atual;
        \item Editar;
        \item Salvar;
        \item Devolver para o centralizador.
    \end{itemize}

\end{document}
